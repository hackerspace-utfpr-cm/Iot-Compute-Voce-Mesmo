
\pagestyle{fancy}
\renewcommand{\chaptermark}[1]{\markboth{#1}{}}
\renewcommand{\sectionmark}[1]{\markright{\thesection\ #1}}
\fancyhf{}
\fancyhead[LE,RO]{\bfseries\thepage}
\fancyhead[LO]{\bfseries\rightmark}
\fancyhead[RE]{\bfseries\leftmark}
\renewcommand{\headrulewidth}{0.5pt}
\renewcommand{\footrulewidth}{0pt}
\addtolength{\headheight}{0.5pt}
\setlength{\footskip}{0in}
\renewcommand{\footruleskip}{0pt}
\fancypagestyle{plain}{%
\fancyhead{}
\renewcommand{\headrulewidth}{0pt}
}
%
%\parindent 0in
\parskip 0.05in
%
\begin{document}
\frontmatter
%
\chapter*{\Huge \center Internet das Coisas }
\thispagestyle{empty}
%{\hspace{0.25in} \includegraphics{./ru_sun.jpg} }
\section*{\center Cloadoaldo Basaglia da Fonseca \\Douglas Lohmann \\Marco Aurélio Graciotto Silva \\Paulo Cesar Gonçalves}
\newpage

% licensa / direitos autorais
\subsection*{\center \normalsize Este trabalho está licenciado sob uma Licença Creative Commons Atribuição 4.0 Internacional. Para ver uma cópia desta licença, visite \\ http://creativecommons.org/licenses/by/4.0/.}

\subsection*{\center \normalsize Os diagramas de projeto foram construídos com o software Fritzing e estão licenciados sob uma Licença Creative Commons Atribuição 4.0 Internacional}

\subsection*{\center \normalsize Este trabalho foi financiado pela Fundação Araucária - Apoio ao Desenvolvimento Científico e Tecnológico do Paraná, por meio do Edital Redes Digitais de Cidadania do Estado do Paraná (Ministério das Comunicações), aprovado em 2013. Foi desenvolvido por alunos e professores da  Universidade Tecnológica Federal do Paraná (UTFPR), campus Campo Mourão}

\subsection*{\center \normalsize Uma versão online desse material está disponível em https://github.com/lohmanndouglas/Iot-Compute-Voce-Mesmo.git }


%

% dedicatoria 
%\chapter*{\center \normalsize To my Son}
%
\tableofcontents
%
\mainmatter

% Objetivos 
\chapter{Uma Visão Geral}
% your text here
Computadores pessoais e smartphones formam uma rede de dispositivos conectados a Internet. A questão agora é permitir  que outros dispositivos tais como relógios, máquinas de lavar, geladeiras e demais objetos do nosso cotidiano possam conectar-se a rede e trocar informações. Esta fase em formação está introduzindo um novo paradigma chamado de Internet das Coisas (do inglês, Internet of Things - IoT), no qual pessoas, animais e coisas do nosso cotidiano estão conectados à rede e interagirem entre si.


A Internet das Coisas mudará tudo, inclusive nós mesmos. Considerando o impacto que Internet já causou na comunicação, nos negócios, na ciência, no governo e na educação, percebemos claramente que a Internet é uma das mais importantes e poderosas invenções de toda a história humana \cite{daveevans2011}. Devido ao desenvolvimento das tecnologias de informação, principalmente da Internet, podemos nos comunicar tranquilamente com qualquer parte do mundo, Desta forma, possuímos a oportunidade de conhecer muitas coisas - novas pessoas, culturas, sistemas políticos, desenvolvimento de cada país e muitas coisas mais - por meio de alguns cliques. 


Podemos dividir a Internet em três fases. A primeira fase é a Internet como uma rede de computadores. Na segunda fase, a Internet pode ser considerada uma rede de pessoas e comunidades e atualmente estamos vivendo a evolução para terceira fase, a Internet das Coisas (IoT). Nesta fase a rede passa a interligar vários tipos de objetos e dispositivos inteligentes do nosso cotidiano que vão interagir entre si e conosco \cite{nicbr}.
Segundo \cite{atzori2010internet}, a ideia básica de IoT consiste na presença de uma diversidade de  objetos que interagem e cooperam entre si a fim de atingir um objetivo comum. Para tal compartilham informações utilizando métodos de endereçamento único e protocolos de comunicação padronizados. 

Este material apresenta os principais conceitos relacionados a Internet das Coisas e também apresenta uma atividade prática para implementação de IoT. Os próximos tópicos são referências básicas para a construção da rede de sensores e a comunicação dos sensores com a Internet. 
