%
\chapter{Infraestrutura da Internet das Coisas}


\section{Componentes envolvidos}

\subsection{O cérebro}

O Arduino foi a escolha para esse projeto, pois tem consumo de energia baixo, é fácil de programar,muitas bibliotecas já estão implementadas e são livres para uso, além disso, é um hardware livre. 

Apesar de não ter um poder computacional muito alto, o Arduino é ideal para uso nesse projeto. Além disso, ele tem pinos (22 no total) que são ideais para conectar sensores e botões. Outra característica é que o Arduino não possui sistema operacional.

\subsection{O rádio}

Com a necessidade de comunicação e coleta de dados dos sensores e devido a possível distância entre eles, é necessário uma conexão sem fio. Para tais fins, é utilizado um pequeno rádio, nesse caso, o modelo nRF24l01, que tem baixo consumo de bateria e baixo preço.

\subsection{Software}

Além dos componentes de hardware, vamos utilizar alguns softwares, na maior parte bibliotecas com alguns exemplos de sensores já implementados. 

\section{Construção da Rede de Sensores e Atuadores}

Utilizando a biblioteca e as plataformas descritas no Capítulo~\ref{MateSof}, é possível configurar uma infraestrutura para a rede IoT, como ilustra a \ref{fig:iot}. A rede implementada possui
basicamente três componentes:
controlador, gateway e nós finais, que são nós sensores e atuadores. Os nós sensores e atuadores são responsáveis
pela interação com o ambiente, seja pela coleta de informações por meio de sensores, emissão de sinais de alerta ou ativação de certos dispositivos por meio de atuadores, por
exemplo, o ar condicionado ou uma lâmpada.

Os principais componentes de uma rede de sensores são apresentados nas próximas subseções.

\begin{figure}[H]
      \centering
      \includegraphics[scale=0.50]{figuras/Diagrama.png}
      \caption{Infraestrutura iot}
      \label{fig:iot}
\end{figure}

A Figura~\ref{fig:iot} ilustra a infraestrutura de uma rede de sensores e atuadores, pode-se perceber a presença do nó gateway conectando os nós sensores e o controlador. Embora na imagem a ligação entre os dispositivos é representada por linhas, na prática a ligação entre os nós na maioria das vezes não é feita por jumper e sim via rádio frequência. 

\subsection{Nó sensor ou atuador}
	Esse nó realiza a leitura de sensores e pode também funcionar como um nó atuador, enviando e recebendo dados do gateway. Esse nó pode funcionar em modo Sleep para economizar bateria.

\subsection{Nó repetidor}
	Esse nó só é necessário quando os nós sensores e gateway não conseguem se comunicar, devido à distância em que estão localizados, por esse motivo não está representado na Figura~\ref{fig:iot}. Esse nó tem como função repetir as dados para outros nós a fim de aumentar a distância de comunicação entre os nós. Em muitas aplicações esse nó não está presente.

\subsection{Nó gateway}

O gateway atua na ligação entre o controlador e rede de rádios. Ele traduz as mensagens do rádio para um protocolo que pode ser entendido por um controlador. 

Na biblioteca MySensors existem três implementações de Gateway: Ethernet Gateway, SerialGateway e MQTTGateway. Nesse material, vamos utilizar o SerialGateway pela facilidade de implementação e, também, pela compatibilidade com o controlador escolhido.
	
\subsection{Controlador}

O controlador pode realizar as seguintes funções:
\begin{itemize}
    \item Enviar parâmetros de configuração para os sensores na rede de rádio (tempo e identificadores de sensores únicos);
    \item Acompanhar os dados mais recentes enviados pelos sensores e atuadores;
    \item Fornecer informações de status de volta para sensores e atuadores, por exemplo, o estado atual (on / off / loadLevel) para uma luz;
    \item Fornecer controles de interface do usuário para atuadores;
    \item Executar horários predefinidos ou cenas, por exemplo, ao pôr do sol acender as luzes do jardim.
\end{itemize}
    
Neste material, vamos utilizar como controlador o Raspberry e o framework Pimatic.

\section{Entendendo o protocolo serial MySensors versão 1.5}

O protocolo utilizado para comunicação entre o serial gateway e controlador consiste de mensagens textuais em que cada dado é separado por ponto e virgula (;) e quebra de linha no final da mensagem. Sendo assim, a mensagem possui a seguinte estrutura:

%colocar uma caixa/imagem

 \vbox{
\noindent \rule{11.1cm}{0.4pt}\par
\noindent node-id; child-sensor-id; message-type; ack; sub-type; payload;\par%\vspace{-0.66\baselineskip}
\noindent \rule[0.90\baselineskip]{11.1cm}{0.6pt}
}



\begin{description}
  \item[node-id] identificação exclusiva do nó que envia ou deve receber a mensagem (endereço);
  \item[child-sensor-id] Cada nó pode ter vários sensores ligados, esse campo identifica qual é o sensor “child” do nó;
  \item[message-type] Tipo de mensagem enviada.
  \item[ack] Outgoing: 0 = mensagem não reconhecida, 1 = pedido ack do nó de destino; Incoming: 0 = mensagem normal, 1 = esta é uma mensagem de ack;
  \item[sub-type] Dependendo da messageType, este campo tem um significado diferente.
  \item[payload] Carga útil (max 25 bytes).
  
\end{description}

\section{Códigos Mysensors}

Nesta seção é apresentado um exemplo de código da biblioteca MySensors. Esse código apresenta os principais métodos utilizados na maioria dos exemplos. 

\lstinputlisting[language=, caption={Código Mysensors}]{code/exemplo.ino}

%colocar srccode{} nas funções 

Para iniciar, precisamos criar uma instância da biblioteca MySensors e depois ativá-la com a instrução gw.begin(). Na primeira vez que o nó é ligado, o controlador atribui um id único a ele. Esse id é salvo na memória EEPROM do Arduino. Caso ele seja religado ou resetado, o id é automaticamente resgatado. O sensor deve ser apresentado para o controlador. Para isso, é utilizado a função gw.present(child-sensor-id, sensor-type).

Para enviar uma mensagem deve-se criar um container MyMessage usando a função msg(child-sensor-id, variable-type). No escopo da função loop, a mensagem é enviada com o método send(msg.set(payload)).

\section{Controlador Pimatic}

Para as atividades desenvolvidas nesse material utilizamos o 
controlador Pimatic. 

Pimatic~\footnote{http://www.mysensors.org/controller/pimatic} é um framework de automação residencial que é executado no Node.js. Ele fornece uma plataforma extensível comum para controle de casa e tarefas de automação
\cite{pimatic}.

A configuração do Pimatic é realizada por meio de somente um arquivo, denominado config.json. Esse arquivo está no formato JSON e é dividido em quatro seções: Configurações (settings), Plugins, Dispositivos (devices) e Regras (Rules). A seguir, exemplos das quatro seções são apresentados.

